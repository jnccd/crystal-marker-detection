%%% KCSS Dissertation Template
%%%
%%% Take care of all fields marked <++FOO++>.
%%% When using Vim LaTeX Suite <http://vim-latex.sourceforge.net/>,
%%% then Ctrl-j should navigate to the next such field.

\documentclass[10pt]{book}
\usepackage[language=english,theorems=numbersfirst,paper=a4paper]{ifiseries}
\usepackage{listings}
\usepackage[ruled,vlined]{algorithm2e}
\usepackage{pdfpages}
\usepackage{todonotes}
\usepackage{breakurl}
\usepackage{subcaption}
\usepackage{hyperref}
\usepackage{lipsum}
\usepackage{parcolumns}
\setlength{\marginparwidth}{3cm}
\addbibresource{thesis.bib}

\graphicspath{{./resources/}}

\newenvironment{localsize}[1]
{%
  \clearpage
  \let\orignewcommand\newcommand
  \let\newcommand\renewcommand
  \makeatletter
  \input{bk#1.clo}%
  \makeatother
  \let\newcommand\orignewcommand
}
{%
  \clearpage
}

\newcommand{\figureref}[1]{\textbf{\autoref{#1}}}
\newcommand{\figurereft}[2]{\textbf{\autoref{#1} #2}}
\newcommand{\toxdo}[1]{}

\newcommand*\NewPage{\newpage\null\thispagestyle{empty}\newpage}

\begin{document}

\frontmatter

\studtitlepage%
{ArUco Marker Detection on Flexible Photonic Crystals}%
{}
{Niklas Carstensen}%
{Master's Thesis}%
{\today}%
{Prof. Dr. Reinhard Koch}%
{Jakob Nazarenus, Tim Michels}%
\NewPage{}
\setcounter{page}{2}
\studeidesstatt
\NewPage{}

\selectlanguage{english}
\setcounter{page}{3}
\chapter*{Abstract}
TODO

\tableofcontents
\listoffigures
%\listoftodos
\mainmatter
\chapter{Introduction}

The world faces a shortage of health workers that is projected to endure into 2030 \cite{BMJHealthWorkforce22}, while 47\% of available health workers cared for 22\% of the worlds population in 2020. The shortage and unequal distribution of health workers drive the need for more efficient care for patients. 

A part of the solution for this problem could be small and inexpensive testing equipment \cite{Fab23}, which would allow for fast and regular assessment of ones health without requiring the time of healthcare professionals \cite{POC12}. Such an approach is called \ac{POC}, since the testing and evaluation is done by and of the patient. These kits are also called \acp{POCT}. One type of \ac{POCT} are wearables. However traditional wearables are based on electrical readout systems \cite{gao2019flexible}, which necessitate conducting wires, a battery and may be invasive. To overcome the drawbacks of wearables using electrical readout systems, optical readout systems can be used instead \cite{nguyen2021wearable}. One approach for an optical readout system is using commercially available adhesive tape, which acts as a \ac{PCS}, and etching \ac{ArUco} fiducial markers into it \cite{Fab23}. This is done by first applying a negative photoresist to the tape and then shining UV light on it, while masking the parts of it that are supposed to become part of the \ac{ArUco} shape later. Then an \ac{IBE} process is applied to the tape and finally the heat resist of the first step is removed during a ultrasonic heat bath.

\section{Problem Statement}

This work focuses on the detection of the \ac{ArUco} fiducial markers that were etched into tape using neural networks. Since the tapes should be used as \acp{POCT} by patients, the technology developed here is supposed to run on patients smart phones. While smart phones are easily carryable, they come with the drawback of not having much space or energy for large \acp{GPU}, which limits the choice of networks.

\section{Outline}

TODO

\chapter{Preliminaries}
\label{chap:prelim}

\section{Definitions}

\section{Neuronal Networks}

\section{CNNs}

\section{Python}

\section{Shapely}

\section{Open CV}

\subsection{Classic ArUco Detection}

\section{Albumentations}

\section{Pytorch}

\section{HyperOpt}

\section{mAP Scores}

\subsection{Pascal Voc 2007}

\subsection{Pascal Voc 2010}

\subsection{COCO}

\chapter{Related Work}
\label{chap:relatedw}

\section{ArUco Marker Detection}

\section{Training Image Augmentations}

\section{Classic Detection Approaches?}

\chapter{Networks}
\label{chap:netw}

\section{Yolov5}

\section{Yolov8}

\section{YolovNAS}

\section{MobileNetV2}

\section{ViTs}

\chapter{Implementation}
\label{chap:implement}

\section{Used Hardware}

\section{Automated Annotations}

\section{Software Architecture}

\section{Augmentations}

\section{Synthetic Data}

\section{Plotting}

\section{Automated Evaluation of Multiple Ensemble Trainings}

\section{Hyperparameter Optimization}

\section{Point Detection}

\chapter{Evaluation}
\label{chap:eval}

\section{Test Data}

\section{Human Testing}

\section{Augmenting from Zero}

\section{Augmenting from Optimum}

\section{Discussion}

%\subsection{Confusion Matrix in Object Detection}

\chapter{Conclusion}
\label{chap:conclusion}

\section{Summary}

\section{Future Work}

\subsection{Deploying on Mobile}

\subsection{Tracking}

\chapter{Appendix}

\section{List of Abbreviations}

\begin{acronym}
% From bachelor thesis
\acro{GUI}{Graphical User Interface}
\acro{UI}{User Interface}
\acro{JSON}{JavaScript Object Notation}
\acro{SVG}{Scalable Vector Graphics}
\acro{LMB}{left mouse button}
\acro{AABB}{Axis Aligned Bounding Box}
\acroplural{AABB}{Axis Aligned Bounding Boxes}
\acro{HTML}{Hypertext Markup Language}
\acro{GIMP}{GNU Image Manipulation Program}

% From Fabio
\acro{POC}{Point-of-care}
\acro{POCT}{Point-of-care Testing Device}
\acro{IBE}{ion beam etching}
\acro{PCS}{Photonic Crystal Slab}

% From master thesis
\acro{ArUco}{Augmented Reality University of Cordoba}
\acro{GPU}{Graphics Processing Unit}

\end{acronym}

\section{User Evaluation Files}


\backmatter
\tocbibliography

\end{document}