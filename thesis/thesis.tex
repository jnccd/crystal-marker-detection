%%% KCSS Dissertation Template
%%%
%%% Take care of all fields marked <++FOO++>.
%%% When using Vim LaTeX Suite <http://vim-latex.sourceforge.net/>,
%%% then Ctrl-j should navigate to the next such field.

\documentclass[10pt]{book}
\usepackage[language=english,theorems=numbersfirst,paper=a4paper]{ifiseries}
\usepackage{listings}
\usepackage[ruled,vlined]{algorithm2e}
\usepackage{pdfpages}
\usepackage{todonotes}
\usepackage{breakurl}
\usepackage{subcaption}
\usepackage{hyperref}
\usepackage{lipsum}
\usepackage{parcolumns}
\setlength{\marginparwidth}{3cm}
\addbibresource{thesis.bib}

\graphicspath{{./resources/}}

\newenvironment{localsize}[1]
{%
  \clearpage
  \let\orignewcommand\newcommand
  \let\newcommand\renewcommand
  \makeatletter
  \input{bk#1.clo}%
  \makeatother
  \let\newcommand\orignewcommand
}
{%
  \clearpage
}

\newcommand{\figureref}[1]{\textbf{\autoref{#1}}}
\newcommand{\figurereft}[2]{\textbf{\autoref{#1} #2}}
\newcommand{\toxdo}[1]{}

\newcommand*\NewPage{\newpage\null\thispagestyle{empty}\newpage}

\begin{document}

\frontmatter

\studtitlepage%
{ArUco Marker Detection on Flexible Photonic Crystals}%
{}
{Niklas Carstensen}%
{Master's Thesis}%
{\today}%
{Prof. Dr. Reinhard Koch}%
{Jakob Nazarenus, Tim Michels}%
\NewPage{}
\setcounter{page}{2}
\studeidesstatt
\NewPage{}

\selectlanguage{english}
\setcounter{page}{3}
\chapter*{Abstract}
TODO

\tableofcontents
\listoffigures
%\listoftodos
\mainmatter
\chapter{Introduction}

\section{Problem Statement}

\section{Outline}

\chapter{Preliminaries}
\label{chap:prelim}

\section{Definitions}

\section{Neuronal Networks}

\section{CNNs}

\section{Python}

\section{Shapely}

\section{Open CV}

\subsection{Classic ArUco Detection}

\section{Albumentations}

\section{Pytorch}

\section{HyperOpt}

\section{mAP Scores}

\subsection{Pascal Voc 2007}

\subsection{Pascal Voc 2010}

\subsection{COCO}

\section{Used Hardware}

\chapter{Related Work}
\label{chap:relatedw}

\section{ArUco Marker Detection under Occlusion}

\section{Dark ChArUco Marker Pose Estimation}

\section{A Novel Marker Detection System for People with Visual Impairment Using the Improved Tiny-YOLOv3 Model}

\section{Underwater Marker-Based Pose-Estimation With Associated Uncertainty}

\section{A Practical Framework for the Development of Augmented Reality
Applications by using ArUco Markers}

\chapter{Networks}
\label{chap:netw}

\section{Yolov5}

\section{Yolov8}

\section{YolovNAS}

\section{MobileNetV2}

\section{ViTs}

\chapter{Implementation}
\label{chap:implement}

\section{Automated Annotations}

\section{Software Architecture}

\section{Augmentations}

\section{Synthetic Data}

\section{Automated Evaluation of Multiple Ensemble Trainings}

\section{Hyperparameter Optimization}

\section{Point Detection}

\chapter{Evaluation}
\label{chap:eval}

\section{Test Data}

\section{Human Testing}

\section{Augmenting from Zero}

\section{Augmenting from Optimum}

%\subsection{Confusion Matrix in Object Detection}

\chapter{Conclusion}
\label{chap:conclusion}

\section{Summary}

\section{Future Work}

\chapter{Appendix}

\section{List of Abbreviations}

\begin{acronym}
\acro{KIELER}{Kiel Integrated Environment for Layout Eclipse Rich Client}
\acro{KEITH}{KIEL Environment Integrated in Theia}
\acro{LSP}{Language Server Protocol}
\acro{IDE}{Integrated Development Environment}
\acro{ELK}{Eclipse Layout Kernel}
\acro{GUI}{Graphical User Interface}
\acro{UI}{User Interface}
\acro{JSON}{JavaScript Object Notation}
\acro{JSON-RPC}{JavaScript Object Notation - Remote Procedure Call}
\acro{npm}{Node Package Manager}
\acro{SVG}{Scalable Vector Graphics}
\acro{UX}{User Experience}
\acro{MDE}{Model Driven Development}
\acro{KLighD}{Kieler Light Weight Diagrams}
\acro{SCCharts}{Sequential Constructive Statecharts}
\acro{melk}{ELK Metadata File}
\acro{LMB}{left mouse button}
\acro{AABB}{Axis Aligned Bounding Box}
\acroplural{AABB}{Axis Aligned Bounding Boxes}
\acro{HTML}{Hypertext Markup Language}
\acro{GIMP}{GNU Image Manipulation Program}
\end{acronym}

\section{User Evaluation Files}


\backmatter
\tocbibliography

\end{document}